\documentclass[a4paper,11pt]{scrartcl}

\usepackage{graphicx}
\usepackage[utf8]{inputenc} 
\usepackage{amsmath,amssymb,amsthm} 
\usepackage[round]{natbib}
\usepackage{url}
\usepackage{xspace}
\usepackage[left=20mm,top=20mm]{geometry}
\usepackage{algorithmic}
\usepackage{subcaption}
\usepackage{mathpazo}
\usepackage{booktabs}
\usepackage{hyperref}
\hypersetup{
    colorlinks=true,
    linkcolor=black,
    filecolor=magenta,      
    urlcolor=cyan,
    citecolor=black,
}


\newcommand{\ie}{i.e.}
\newcommand{\eg}{e.g.}
\newcommand{\reffig}[1]{Figure~\ref{#1}}
\newcommand{\refsec}[1]{Section~\ref{#1}}

\setcapindent{1em} %-- for captions of Figures
\hbadness=10000 %-- set tolerance for under-full box
\hfuzz=10000pt %-- set tolerance for overfull box

\renewcommand{\algorithmicrequire}{\textbf{Input:}}
\renewcommand{\algorithmicensure}{\textbf{Output:}}


\title{Research Area}
\author{Odunayo Rotimi.}
\date{\today}

\begin{document}
\maketitle

\section*{Research outlines}
\begin{itemize}
    \item This thesis shall also inquire into the possibility of transforming financial information into information-flow graphs combined with Natural Language Processing techniques to study the inter-relatedness of investors' perception about trade information and how they drive resultant investment decisions. We shall then delve into predicting the inter-relatedness of the influence of trading news and possible dependence investment decisions on key sources or influencers using the emergent  Graph Convolutional Networks (GCN), designed to capture relational information. Performance will be  measured against that of other binary classification Machine Learning algorithms.
          Such transformation is well known in genetic modelling, communications and social networking. \label{obj_1}
    \item The thesis aims to extend the novelty in developing mixed or hybrid Deep learning models that are viable for the prediction of derivatives. Method to be considered include the transformation of of financial data from 1-dimensional array to image-like 2-dimensional array in an effort to reduce associated dimensionality problems, combined with a parameterized version Neural Network (NN). \label{obj_2}
    \item Lastly, this research work extend the study of application of Generative Adversarial Network (GAN) to crypto-securities to contribute to the state of research. This aspect would detail a systematic enquiry comparative forecasting advantage the GAN may posses over other Deep Learning models.\label{obj_3}
\end{itemize}

\end{document}